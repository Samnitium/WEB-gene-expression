\documentclass{article}
\title{WEB-gene-expression Documentation}
\author{Chiara Bartalotta  \\
	Universit\'a degli Studi Roma Tre  \\
	\and 
	Davide Bernardini \\
	Universit\'a degli Studi Roma Tre \\
	\and 
	Dario Santilli \\
	Universit\'a degli Studi Roma Tre \\
	}

\date{July 2014  }
% Hint: \title{what ever}, \author{who care} and \date{when ever} could stand 
% before or after the \begin{document} command 
% BUT the \maketitle command MUST come AFTER the \begin{document} command! 
\begin{document}

\maketitle


\begin{abstract}
This paper is a documentation that aims at showing the implementation details f WEB-gene-expression platform. The first paragraph discusses about the type of user which will interact with the platform. The technological choices, to develop the client-side and the server-side, are motivate in the second paragraph. The third paragraph explains how the project code is divided and shows the implementation details.


\section{Introduction}
WEB-gene-expression is a web platform to manage gene expression data. Through gene expression process, information from a gene is converted in a functional macromolecule, often proteins. WEB-gene-expression platform allows the user to visualize experiments done on gene expression data.

\section{User}





\section{Documentclasses} \label{documentclasses}

\begin{itemize}
\item article
\item book 
\item report 
\item letter 
\end{itemize}


\begin{enumerate}
\item article
\item book 
\item report 
\item letter 
\end{enumerate}

\begin{description}
\item[article\label{article}]{Article is \ldots}
\item[book\label{book}]{The book class \ldots}
\item[report\label{report}]{Report gives you \ldots}
\item[letter\label{letter}]{If you want to write a letter.}
\end{description}


\section{Conclusions}\label{conclusions}
There is no longer \LaTeX{} example which was written by \cite{doe}.


\begin{thebibliography}{9}
\bibitem[Doe]{doe} \emph{First and last \LaTeX{} example.},
John Doe 50 B.C. 
\end{thebibliography}

\end{document}
