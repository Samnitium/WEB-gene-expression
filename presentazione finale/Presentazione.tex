\documentclass[hyperref={pdfpagelabels=false}]{beamer}
\usepackage{lmodern}
\usetheme{CambridgeUS}
\title{\\Meccanismi che condizionano l'espressione genica\\}  
\author{\\ C. Bartalotta, D. Santilli e D. Bernardini} 
\date{\today} 
\begin{document}
\logo{\includegraphics[scale=0.05]{logoRomaTre}}


\begin{frame}
\titlepage
\end{frame} 


\begin{frame}\frametitle{Titolo1}
\textbf{Espressione genica:}\\
processo grazie al quale l'informazione contenuta in un gene viene convertita in una macromolecola funzionale.
\'E un processo a pi\'u stadi:\pause 
\begin{enumerate}
\item trascrizione del DNA in RNA:\\
produzione dell'RNA complementare a una delle due eliche del gene  \pause 
\item maturazione:\\
l'RNA trascritto puo' essere modificato (maturazione) per dare origine a quello che si chiama RNA (funzionante) maturo \pause 
\item traduzione:\\
l'RNA messaggero o mRNA viene tradotto nella proteina corrispondente
\end{enumerate}
\end{frame}


\begin{frame}\frametitle{Controllo dell'espressione genica}
L'evoluzione ha consentito che le funzioni cellulari si accendano e spengano a seconda del loro effettivo bisogno. Questo f\'a si che le cellulule si differenzino al fine di svolgere ruoli specializzati.
\\
Due tipi di gene:
\begin{itemize}
\item \textbf{Geni regolati:}  caratterizzati da espressione regolata per far si che la quantit\'a del loro prodotto (proteina o RNA) sia controllata in base al fabbisogno cellulare. I geni regolati sono quindi espressi solo in alcuni tipi di cellule e solo in determinati momenti, in risposta a determinati stimoli.
\item \textbf{Geni consecutivi:} costantemente attivi perchè essenziali per svolegere processi di base. Questo tipo di gene, anche chiamato \emph{housekeeping genes}, vengono continuamente trascritti in quanto codificano proteine necessarie per il mantenimento di funzioni generali. Essi sono riconosciuti da attivatori  presenti in tutte le 
cellule.
\end{itemize}
\end{frame}
\begin{frame}\frametitle{Regolazione dell'espressione genica}
La regolazione pu\'o avvenire a diversi livelli:
\begin{block}{Livello di trascrizione}
Le proteine regolatrici, attivatori o repressori, interagiscono con il DNA ed in base ai cambiamenti ambientali attivano o disattivano i geni. A questo livello, il controllo stabilisce quali geni vengono trascritti e per quanto tempo. Questo livello di regolazione costituisce quindi un ruolo fondamentale in quanto, determinando quali geni devono essere attivati o spenti, si stabilisce anche quali proteine verranno sintetizzate nell'arco di vita cellulare.
\end{block}
\begin{block}{Livello di maturazione}
I controlli a livello di maturazione agiscono determinando quali parti dei trascritti primari entrano a far parte degli mRNA cellulari.
\end{block}
\begin{block}{Livello di traduzione}
I controlli a livello di traduzione determinano se un mRNA deve essere tradotto, per quanto tempo e quanto frequentemente.
\end{block}
\end{frame}


\begin{frame}\frametitle{Titolo2}
\begin{itemize}
\item \textbf{Geni regolati:}  caratterizzati da espressione regolata per far si che la quantit\'a del loro prodotto (proteina o RNA) sia controllata in base al fabbisogno cellulare.
\item \textbf{Geni consecutivi:} costantemente attivi perchè essenziali per svolegere processi di base.
\end{itemize}
\end{frame}


\begin{frame}\frametitle{Regolazione dell'espressione genica}
La regolazione pu\'o avvenire a diversi livelli:
\begin{block}{Livello di trascrizione}
Le proteine regolatrici, attivatori o repressori, interagiscono con il DNA ed in base ai cambiamenti ambientali attivano o disattivano i geni. A questo livello, il controllo stabilisce quali geni vengono trascritti e per quanto tempo.
\end{block}
\begin{block}{Livello di maturazione}
I controlli a livello di maturazione agiscono determinando quali parti dei trascritti primari entrano a far parte degli mRNA cellulari.
\end{block}
\begin{block}{Livello di traduzione}
I controlli a livello di traduzione determinano se un mRNA deve essere tradotto, per quanto tempo e quanto frequentemente.
\end{block}
\end{frame}


\begin{frame} \frametitle{Controllo a livello della trascrizione}
\begin{block}{Blocco1}
\begin{itemize}
\item 
\item 
\item 
\end{itemize}
\end{block}
\begin{exampleblock}{Blocco2}
\begin{itemize}
\item 
\item 
\end{itemize}
\end{exampleblock}
\bigskip
\bigskip
\bigskip
\end{frame}
\end{document}