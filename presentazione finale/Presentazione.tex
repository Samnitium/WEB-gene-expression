\documentclass[hyperref={pdfpagelabels=false}]{beamer}
\usepackage{lmodern}
\usetheme{CambridgeUS}

\title{\\Meccanismi che condizionano l'espressione genica\\}  
\author{\\ C. Bartalotta, D. Santilli e D. Bernardini} 
\date{\today} 
\begin{document}
\logo{\includegraphics[scale=0.05]{logoRomaTre}}
\begin{frame}
\titlepage
\end{frame} 

\begin{frame}\frametitle{Titolo1}
\textbf{Espressione genica:}\\
processo grazie al quale l'informazione contenuta in un gene viene convertita in una macromolecola funzionale.
\'E un processo a pi\'u stadi:\pause 
\begin{enumerate}
\item trascrizione del DNA in RNA:\\
produzione dell'RNA complementare a una delle due eliche del gene  \pause 
\item maturazione:\\
l'RNA trascritto puo' essere modificato (maturazione) per dare origine a quello che si chiama RNA (funzionante) maturo \pause 
\item traduzione:\\
l'RNA messaggero o mRNA viene tradotto nella proteina corrispondente
\end{enumerate}
\end{frame}

\begin{frame}\frametitle{Titolo2}
\begin{itemize}
\item 
\item 
\item 
\end{itemize}
\end{frame}

\begin{frame}\frametitle{Titolo3}
\begin{block}{Blocco1}
\begin{itemize}
\item 
\item 
\item 
\end{itemize}
\end{block}
\begin{exampleblock}{Blocco2}
\begin{itemize}
\item 
\item 
\end{itemize}
\end{exampleblock}
\bigskip
\bigskip
\bigskip
\end{frame}

\end{document}

